\documentclass[a4paper,oneside]{article}
\usepackage[utf8]{inputenc}
\usepackage{amsmath}
\usepackage{amsthm}
\usepackage{amssymb}
\usepackage[top=2cm]{geometry}

\theoremstyle{definition} \newtheorem{exercise}{Exercise}[section]

\author{Sebastiano Tronto (\texttt{sebastiano.tronto@uni.lu})}
\title{Elementary Logic exercises (Prep Camp 2020)}

\begin{document}
\maketitle

\section{Logical operations}

\begin{exercise}
  Determine if the following statements are \textbf{true} or \textbf{false}:
  \begin{enumerate}
    \item ``Today is Tuesday or Germany has more inhabitants than Luxembourg''
    \item ``$7$ is odd and $2+2=5$''
    \item Every number of the form $2^{2^n}+1$, for $n=1,2,3...$, is prime.
  \end{enumerate}
\end{exercise}
\begin{proof}[Solution]
  \begin{enumerate}
    \item \textbf{True}: regardless of when you solve this exercise, Germany
          has more inhabitants than Luxembourg.
    \item \textbf{True}
    \item \textbf{False}: the number $2^{2^5}+1=4294967297=641\times 6700417$
          is not prime.
  \end{enumerate}
\end{proof}

\begin{exercise}
  What is the negation of the sentence ``\emph{I payed attention in class and I
  did not do my homework}'' ?
\end{exercise}
\begin{proof}[Solution]
  ``\emph{I did not pay attention in class \textbf{or} I did my homework}''
\end{proof}

\begin{exercise}
  Simplify the following logical expressions using the properties of logical
  operations (where $A,B$ and $C$ are statements):
  \begin{enumerate}
    \item $A\land(A\lor B)$
    \item $A\lor (B\land A)$
    \item $(A\lor B) \land \neg A$
    \item $A \lor (\neg A\land B)$
    \item $(\neg (A\lor \neg B))\land ((A\lor C) \land \neg C)$
  \end{enumerate}
\end{exercise}
\begin{proof}[Solution]
  They are equivalent to the following (you can check with truth tables):
  \begin{enumerate}
    \item $A$
    \item $A$
    \item $B\land \neg A$
    \item $A\lor B$
    \item Let's do this one in more steps:
      \begin{align*}
        (\neg (A\lor \neg B))\land ((A\lor C) \land \neg C)=
        &(\neg A\land B)\land ((A\lor C)\land \neg C)=\\
        =&(\neg A\land B)\land ((A\land \neg C)\lor (C\land \neg C))=\\
        =&(\neg A\land B)\land ((A\land \neg C)\lor \textbf{false})=\\
        =&(\neg A\land B)\land (A\land \neg C)=\\
        =&A\land \neg A\land B\land \neg C=\\
        =&\textbf{false}
      \end{align*}
  \end{enumerate}
\end{proof}

\section{Implication}

\begin{exercise}
  Fill in the following truth table:
  \begin{align*}
    \begin{array}{|c|c|c|c|c|}
      \hline
      A & B & C & \neg(A\implies B) & (A\implies B) \implies C \\
      \hline
      0 & 0 & 0 & 0 & 0 \\
      \hline
      0 & 0 & 1 & 0 & 1 \\
      \hline
      0 & 1 & 0 & 0 & 0 \\
      \hline
      0 & 1 & 1 & 0 & 1 \\
      \hline
      1 & 0 & 0 & 1 & 1 \\
      \hline
      1 & 0 & 1 & 1 & 1 \\
      \hline
      1 & 1 & 0 & 0 & 0 \\
      \hline
      1 & 1 & 1 & 0 & 1 \\
      \hline
    \end{array}
  \end{align*}
\end{exercise}

\begin{exercise}[Transitivity]
  Prove that the following statement is true for any statements $A,B$ and $C$:
  \begin{align*}
    ((A\implies B)\land (B\implies C))\implies (A\implies C)
  \end{align*}
\end{exercise}
\begin{proof}[Solution]
  Let's rewrite the first part in terms of basic logical operations:
  \begin{align*}
    (A\implies B)\land (B\implies C)=&(B\lor \neg A)\land(C\lor \neg B)
  \end{align*}
  now we can write a truth table for the two parts
  \begin{align*}
    \begin{array}{|c|c|c|c|c|}
      \hline
      A & B & C & (B\lor\neg A)\land(C\lor\neg B) & A\implies C \\
      \hline
      0 & 0 & 0 & 1 & 1 \\
      \hline
      0 & 0 & 1 & 1 & 1 \\
      \hline
      0 & 1 & 0 & 0 & 1 \\
      \hline
      0 & 1 & 1 & 1 & 1 \\
      \hline
      1 & 0 & 0 & 0 & 0 \\
      \hline
      1 & 0 & 1 & 0 & 1 \\
      \hline
      1 & 1 & 0 & 0 & 0 \\
      \hline
      1 & 1 & 1 & 1 & 1 \\
      \hline
    \end{array}
  \end{align*}
  With the help of the truth table we see that whenever the first part
  ``$(\neg (A\lor \neg B))\land ((A\lor C) \land \neg C)$'' is true, also the
  implication ``$A\implies C$'' is true. This shows that the ``big
  implication'' is true. (If you are not convinced, you can add more details to
  this proof, for example by writing more truth tables.)
\end{proof}

\begin{exercise}
What is the contrapositive of ``\emph{If this table is not reserved, we sit
here}'' ?
\end{exercise}
\begin{proof}[Solution]
  ``\emph{If we do not sit here, this table is reserved}''. One could also say
  this in another way, for example ``\emph{We do not sit here because this
  table is reserved}''.
\end{proof}

\section{Quantifiers}

\begin{exercise}
  Write the negation of the following statements:
  \begin{enumerate}
    \item $\exists x\in \mathbb N,\, x^2-2=0$
    \item ``Every prime number is odd''
    \item ``Every person I have met likes pizza''
    \item ``There is at least one number greater than $7$''
    \item $\forall x\in \mathbb N,\,x\geq 0$
    \item $\forall x\in \mathbb Z,\,(\exists y\in\mathbb Z,\,x+y=0)$
  \end{enumerate}
\end{exercise}
\begin{proof}[Solution]
  \begin{enumerate}
    \item $\forall x\in\mathbb N,\, x^2-2\neq 0$
    \item ``There is at least one prime number which is even''
    \item ``I have met at least one person that does not like pizza''
    \item ``Every number is less or equal than $7$''
    \item $\exists x\in\mathbb N,\, x<0$
    \item $\exists x\in \mathbb Z,\, (\forall y\in \mathbb Z,\, x+y\neq 0)$
  \end{enumerate}
\end{proof}

\begin{exercise}
  There is another quantifier that we did not cover in the lecture, namely
  $\exists!$ (read ``there exists exactly one''). For example, the sentence
  ``\emph{there exists exactly one natural number x such that x+2=5}'' can be
  written in symbols as ``$\exists!x\in \mathbb N,\,x+2=5$''.

  In this exercise, your task is to give a formal definition of this quantifier
  using the logical symbols that we have defined in class. In particular, you
  will need the following:
  \begin{itemize}
    \item the universal ($\forall$) and existential ($\exists$) quantifiers
    \item the conjunction $\land$
    \item the implication $\implies$
  \end{itemize}
  Moreover, you will need the equality symbol $=$ between two elements of a set
  (if $a$ and $b$ are two elements of the same set, ``$a=b$'' is a mathematical
  statement and it is \textbf{true} if and only if $a$ and $b$ are the same
  element).
  
  \emph{Warning: your definition must depend on a set $S$ and on a ``variable
  statement'' $A(x)$, as the existential and universal quantifiers.}
\end{exercise}
\begin{proof}[Solution]
  The idea is that we want to write ``\emph{there is $x\in S$ such that $A(x)$
  is true, \textbf{and}, for any other $y\in S$, $A(y)$ is false}. From this 
  we see that the structure of the statement is
  \begin{align*}
    \exists x\in S,\,(\text{``something''}\land\text{``something else''})
  \end{align*}
  The ``something'' part is just $A(x)$. The ``something else'' part can be
  written in different ways, for example
  \begin{align*}
    \forall y\in S,\,(y\neq x\implies \neg A(y)) \quad \text{or}
    \quad \forall y\in S,\,(A(y)\implies y=x)
  \end{align*}
  (notice that the two implications above are one the contrapositive of the
  other); or also
  \begin{align*}
    \forall y\in S\setminus \{x\},\, \neg A(y)
  \end{align*}
  So in conclusion, one way to define ``$\exists!$'' is the following:
  \begin{align*}
    \exists!x\in S,\,A(x)\quad:=\quad\exists x\in S,\,(A(x)\land
    (\forall y\in S,\,(A(y)\implies y=x)))
  \end{align*}
    
\end{proof}

\section{Proofs}

\begin{exercise}
  Prove by induction that
  \begin{align*}
    \forall n\in\mathbb N,\quad \sum_{k=1}^n(2k-1)=n^2
  \end{align*}
    (here $\sum_{k=1}^n(2k-1)$ means $1+3+5+\cdots+ (2n-1)$).
\end{exercise}
\begin{proof}[Solution]
  \textbf{Base case:} for $n=0$ the sum is empty, so we have $0=0$ which is
  true. (If you do not think that the formula makes sense for $n=0$, you can do
  prove it for $n\geq 1$ and fo $n=1$ as a base case.)

  \textbf{Inductive step:} we can assume that the formula works for a generic
  (but fixed) $n\in\mathbb N$ and prove that then it also works for $n+1$. So:
  \begin{align*}
    \sum_{k=1}^{n+1}(2k-1)&=\left(\sum_{k=1}^{n}(2k-1)\right)+2(n+1)-1=\\
                          &=n^2 +2(n+1)-1=\\
                          &=n^2+2n+1=\\
                          &=(n+1)^2
  \end{align*}
  which is the formula for $n+1$.
\end{proof}

\begin{exercise}
  If $n\in \mathbb N$ the \emph{factorial} of $n$, denoted by $n!$ is defined
  as follows:
  \begin{align*}
    n!=\begin{cases}
      1&\text{if } n=0,\\
      n\times (n-1)! & \text{if } n> 0.
    \end{cases}
  \end{align*}
  Prove by induction that if $n\geq 4$ then $n!\geq 2^n$.
\end{exercise}
\begin{proof}[Solution]
  \textbf{Base case:} here the base case is $n=4$, and we have
  $4!=24\geq 16=2^4$.

  \textbf{Inductive step:} we can assume that the inequality is true for $n$,
  and prove that it is also true for $n+1$. We have:
  \begin{align*}
    (n+1)!=(n+1)\times n!\geq (n+1)\times 2^n\geq 2^{n+1}
  \end{align*}
\end{proof}

\begin{exercise}
  Is the following statement true or false? Give a proof of your answer.
  \begin{align*}
    \forall n\in \mathbb N,\, n^2 -4n +5>n
  \end{align*}
\end{exercise}
\begin{proof}[Solution]
  The statement is \textbf{false}. Since it starts with a universal
  quantifier, in order to prove that it is false we just need to provide one
  example of $n\in \mathbb N$ which makes it false. In other words, we need to
  prove that
  \begin{align*}
    \exists n\in \mathbb N,\, n^2-4n+5\leq n
  \end{align*}
  and a proof of this fact is very simple: for $n=2$ we have
  $2^2-4\times 2+5=1\leq 2$.
\end{proof}

\begin{exercise}
  Do the last point of Exercise 1.1 again, but this time give a proof of your
  answer.
\end{exercise}
\begin{proof}[Solution]
  Again, since we have to prove that the statement is false, we just need to
  show one counterexample. for example, the number
  $2^{2^5}+1=4294967297=641\times 6700417$ is not prime.
\end{proof}




\end{document}
