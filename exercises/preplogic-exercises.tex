\documentclass[a4paper,oneside]{article}
\usepackage[utf8]{inputenc}
\usepackage{amsmath}
\usepackage{amsthm}
\usepackage{amssymb}
\usepackage[top=2cm]{geometry}

\theoremstyle{definition} \newtheorem{exercise}{Exercise}[section]

\author{Sebastiano Tronto (\texttt{sebastiano.tronto@uni.lu})}
\title{Elementary Logic exercises (Prep Camp 2020)}

\begin{document}
\maketitle

\section{Logical operations}

\begin{exercise}
  Determine if the following statements are \textbf{true} or \textbf{false}:
  \begin{enumerate}
    \item ``Today is Tuesday or Germany has more inhabitants than Luxembourg''
    \item ``$7$ is odd and $2+2=5$''
    \item Every number of the form $2^{2^n}+1$, for $n=1,2,3...$, is prime.
  \end{enumerate}
\end{exercise}

\begin{exercise}
  What is the negation of the sentence ``\emph{I payed attention in class and I
  did not do my homework}'' ?
\end{exercise}

\begin{exercise}
  Simplify the following logical expressions using the properties of logical
  operations (where $A,B$ and $C$ are statements):
  \begin{enumerate}
    \item $A\land(A\lor B)$
    \item $A\lor (B\land A)$
    \item $(A\lor B) \land \neg A$
    \item $A \lor (\neg A\land B)$
    \item $(\neg (A\lor \neg B))\land ((A\lor C) \land \neg C)$
  \end{enumerate}
\end{exercise}

\section{Implication}

\begin{exercise}
  Fill in the following truth table:
  \begin{align*}
    \begin{array}{|c|c|c|c|c|}
      \hline
      A & B & C & \neg(A\implies B) & (A\implies B) \implies C \\
      \hline
      0 & 0 & 0 & & \\
      \hline
      0 & 0 & 1 & & \\
      \hline
      0 & 1 & 0 & & \\
      \hline
      0 & 1 & 1 & & \\
      \hline
      1 & 0 & 0 & & \\
      \hline
      1 & 0 & 1 & & \\
      \hline
      1 & 1 & 0 & & \\
      \hline
      1 & 1 & 1 & & \\
      \hline
    \end{array}
  \end{align*}
\end{exercise}

\begin{exercise}[Transitivity]
  Prove that the following statement is true for any statements $A,B$ and $C$:
  \begin{align*}
    ((A\implies B)\land (B\implies C))\implies (A\implies C)
  \end{align*}
\end{exercise}

\begin{exercise}
What is the contrapositive of ``\emph{If this table is not reserved, we sit
here}'' ?
\end{exercise}

\section{Quantifiers}

\begin{exercise}
  Write the negation of the following statements:
  \begin{enumerate}
    \item $\exists x\in \mathbb N,\, x^2-2=0$
    \item ``Every prime number is odd''
    \item ``Every person I have met likes pizza''
    \item ``There is at least one number greater than $7$''
    \item $\forall x\in \mathbb N,\,x\geq 0$
    \item $\forall x\in \mathbb Z,\,(\exists y\in\mathbb Z,\,x+y=0)$
  \end{enumerate}
\end{exercise}

\begin{exercise}
  There is another quantifier that we did not cover in the lecture, namely
  $\exists!$ (read ``there exists exactly one''). For example, the sentence
  ``\emph{there exists exactly one natural number x such that x+2=5}'' can be
  written in symbols as ``$\exists!x\in \mathbb N,\,x+2=5$''.

  In this exercise, your task is to give a formal definition of this quantifier
  using the logical symbols that we have defined in class. In particular, you
  will need the following:
  \begin{itemize}
    \item the universal ($\forall$) and existential ($\exists$) quantifiers
    \item the conjunction $\land$
    \item the implication $\implies$
  \end{itemize}
  Moreover, you will need the equality symbol $=$ between two elements of a set
  (if $a$ and $b$ are two elements of the same set, ``$a=b$'' is a mathematical
  statement and it is \textbf{true} if and only if $a$ and $b$ are the same
  element).
  
  \emph{Warning: your definition must depend on a set $S$ and on a ``variable
  statement'' $A(x)$, as the existential and universal quantifiers.}
\end{exercise}

\section{Proofs}

\begin{exercise}
  Prove by induction that
  \begin{align*}
    \forall n\in\mathbb N,\quad \sum_{k=1}^n(2k-1)=n^2
  \end{align*}
    (here $\sum_{k=1}^n(2k-1)$ means $1+3+5+\cdots+ (2n-1)$).
\end{exercise}

\begin{exercise}
  If $n\in \mathbb N$ the \emph{factorial} of $n$, denoted by $n!$ is defined
  as follows:
  \begin{align*}
    n!=\begin{cases}
      1&\text{if } n=0,\\
      n\times (n-1)! & \text{if } n> 0.
    \end{cases}
  \end{align*}
  Prove by induction that if $n\geq 4$ then $n!\geq 2^n$.
\end{exercise}


\begin{exercise}
  Is the following statement true or false? Give a proof of your answer.
  \begin{align*}
    \forall n\in \mathbb N,\, n^2 -4n +5>n
  \end{align*}
\end{exercise}

\begin{exercise}
  Do the last point of Exercise 1.1 again, but this time give a proof of your
  answer.
\end{exercise}



\end{document}
